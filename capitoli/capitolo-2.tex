% !TEX encoding = UTF-8
% !TEX TS-program = pdflatex
% !TEX root = ../tesi.tex

%**************************************************************
\chapter{Descrizione dello stage}
\label{cap:descrizione-stage}
%**************************************************************

\intro{In questo capitolo è presente un’analisi preventiva dei rischi che potevano venire riscontrati durante lo svolgimento dello stage, la lista degli obiettivi da raggiungere e la pianificazione delle ore di lavoro.}\\

%**************************************************************

\section{Analisi preventiva dei rischi}

Durante la fase di analisi iniziale sono stati individuati dei possibili rischi che si potranno incontrare durante il percorso di stage. Si è quindi proceduto a elaborare delle possibili soluzioni per far fronte a tali rischi.

\begin{enumerate}
    \item \textbf{Inesperienza tecnologica:} alcune tecnologie incluse nel progetto sono parzialmente o completamente sconosciute.\\
    \textbf{Soluzione:} studio delle tecnologie nelle prime settimane di stage tramite documentazione ufficiale e corsi forniti dall'azienda.
    \item \textbf{Problematiche hardware:} le macchine utilizzate durante l'implementazione possono essere soggette a malfunzionamenti o guasti, con conseguenti perdite di dati.\\
    \textbf{Soluzione:} condivisione giornaliera su una repository del lavoro svolto.
    \item \textbf{Monitoraggio scadenze:} a causa di altri impegni lavorativi. non sarà sempre possibile confrontarsi con il tutor aziendale per aggiornarlo sullo svolgimento del lavoro.\\
    \textbf{Soluzione:} creazione di un foglio Excel condiviso contente un report giornaliero delle attività svolte.
\end{enumerate}

%**************************************************************

\section{Requisiti e obiettivi}
\label{sec:requisiti-obiettivi}
\subsection*{Notazione}
Si farà riferimento ai requisiti secondo le seguenti notazioni:
\begin{itemize}
	\item \textit{O} per i requisiti obbligatori, vincolanti in quanto obiettivo primario richiesto dal committente;
	\item \textit{D} per i requisiti desiderabili, non vincolanti o strettamente necessari,
		  ma dal riconoscibile valore aggiunto;
	\item \textit{F} per i requisiti facoltativi, rappresentanti valore aggiunto non strettamente 
		  competitivo.
\end{itemize}

Le sigle precedentemente indicate saranno seguite da una coppia sequenziale di numeri, identificativo del requisito.

\subsection*{Obiettivi fissati}
Si prevede lo svolgimento dei seguenti obiettivi:
\begin{itemize}
	\item Obbligatori
	\begin{itemize}
		\item \underline{\textit{O01}}: Acquisizione competenze sulle tematiche sopra descritte;
    	\item \underline{\textit{O02}}: Capacità di raggiungere gli obiettivi richiesti in autonomia seguendo il cronoprogramma;
    	\item \underline{\textit{O03}}: Portare a termine le implementazioni previste con una percentuale di superamento pari al
         80\%.
	\end{itemize}
	
	\item Desiderabili 
	\begin{itemize}
		\item \underline{\textit{D01}}: Portare a termine le implementazioni previste con una percentuale di superamento pari al 100\%.
	\end{itemize}
	
	\item Facoltativi
	\begin{itemize}
		\item \underline{\textit{F01}}: Apportare un valore aggiunto al gruppo di lavoro durante le fasi di progettazione delle interfacce.
	\end{itemize} 
\end{itemize}

%**************************************************************

\section{Pianificazione del lavoro}
Lo stage ha una durata di 8 settimane e prevede lo svolgimento di 320 ore effettive di lavoro. É diviso in due parti, il primo mese di studio delle tecnologie e il secondo di progettazione e implementazione delle maschere.\\
La ripartizione settimanale delle attività è la seguente:
\begin{itemize}
        \item \textbf{Prima Settimana (40 ore)}
        \begin{itemize}
            \item Incontro con persone coinvolte nel progetto per discutere i requisiti e le richieste
            relativamente al sistema da sviluppare;
            \item  Presentazione strumenti di lavoro per la condivisione del materiale di studio e per la gestione
            dell’avanzamento;
            \item Condivisione scaletta di argomenti;
            \item Ripasso del linguaggio Java SE;
            \item Ripasso concetti Web (Servlet, servizi Rest, Json ecc.).
        \end{itemize}
        \item \textbf{Seconda Settimana (40 ore)} 
        \begin{itemize}
            \item Studio principi generali di Spring Core (IOC, Dependency Injection);
            \item Studio SpringBoot;
            \item Studio Spring Data/DataRest.
        \end{itemize}
        \item \textbf{Terza Settimana (40 ore)} 
        \begin{itemize}
            \item Ripasso linguaggio Javascript;
            \item Studio del linguaggio TypeScript.
        \end{itemize}
        \item \textbf{Quarta Settimana (40 ore)} 
        \begin{itemize}
            \item Studio piattaforma NodeJS e AngularCLI;
            \item Studio framework Angular.
        \end{itemize}
        \item \textbf{Quinta Settimana (40 ore)} 
        \begin{itemize}
            \item Analisi e studio del progetto ShopChain;
            \item Progettazione ed implementazione della nuova maschera di accesso.
        \end{itemize}
        \item \textbf{Sesta Settimana (40 ore)} 
        \begin{itemize}
            \item Progettazione ed implementazione nuova maschera "Gestione Profilo Venditore";
            \item Scrittura dei service (su front-end) di chiamata al back-end.
        \end{itemize}
        \item \textbf{Settima Settimana (40 ore)} 
        \begin{itemize}
            \item Progettazione ed implementazione nuova maschera "Gestione Profilo Acquirente";
            \item Scrittura dei service (su front-end) di chiamata al back-end.
        \end{itemize}
        \item \textbf{Ottava Settimana - Conclusione (40 ore)} 
        \begin{itemize}
            \item Termine integrazioni e collaudo finale.
        \end{itemize}
\end{itemize}