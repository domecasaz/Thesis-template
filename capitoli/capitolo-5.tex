% !TEX encoding = UTF-8
% !TEX TS-program = pdflatex
% !TEX root = ../tesi.tex

%**************************************************************
\chapter{Verifica e validazione}
\label{cap:verifica-validazione}

\section{Accessibilità}
\subsection{Attributi HTML}

Per rendere l'applicazione web più accessibile sono stati utilizzati diversi attributi HTML specifici per l'accessibilità:
\begin{itemize}
    \item \textbf{scope}: utilizzato per facilitare la lettura delle tabelle agli screen reader;
    \item \textbf{summary}: utilizzato per riassumere il contenuto di una tabella di piccole dimensioni;
    \item \textbf{title}: utilizzato per descrivere gli elementi che altrimenti non sarebbero leggibili dagli screen reader, totalmente o parzialmente;
    \item \textbf{lang}: utilizzato per indicare agli screen reader in che lingua leggere una parola o una frase diversa dall'italiano.
\end{itemize}


\subsection{Colori}
I colori del sito sono stati selezionati per passare il test WCAG AA che richiede un rapporto di contrasti di almeno 4.5:1 per il testo normale e 3:1 per il testo in grassetto. Di seguito è riportata la tabella dei colori del testo e del loro sfondo e il loro rapporto di contrasto.

\begin{longtable}[c]{|l|l|l|}
\caption{Rapporto contrasto testo-sfondo}
\label{tab:contrasto-colori}
\\ \hline
\rowcolor{gray!40}
\textbf{Colore testo} &
\textbf{Colore sfondo} &
\textbf{Rapporto} \\ \hline
\endhead

\#FFFFFF & \#3F51B5 & 6.87:1 \\ \hline

\#000000 & \#FFFFFF & 21:1 \\ \hline

\#000000 & \#F2F2F2 & 18.75:1 \\ \hline

\#FAF200 & \#3F51B5 & 5.81:1 \\ \hline

\#000000 & \#FAF200 & 17.76:1 \\ \hline

\#CF1507 & \#FFFFFF & 5.58:1 \\ \hline
\end{longtable}


\subsubsection{Altri elementi di accessibilità}
\begin{itemize}
    \item Sono stati rimossi i link circolari per evitare confusione negli utenti e permettere di identificare univocamente il percorso raggiunto;
    \item Si sono evitati testi scorrevoli, lampeggianti, barrati ed in generale font troppo elaborati per agevolare la lettura;
    \item Sono stati utilizzati gli attributi \gls{WAI-ARIA}\glsfirstoccur per specificare meglio alcuni elementi:
    \begin{itemize}
        \item \textbf{role}: usato per definire la funzione di alcuni tag;
        \item \textbf{aria-hidden}: usato per nascondere elementi inutili agli screen reader perché già presenti descrizioni testuali.
    \end{itemize}
\end{itemize}

%**************************************************************

\section{Validazione e collaudo}
É stata organizzata settimanalmente una riunione di allineamento con tutte le persone coinvolte nel progetto. In queste riunione si otteneva un feedback sul lavoro svolto fino a quel momento e una lista dei possibili cambiamenti da fare per raggiungere l’obiettivo voluto.\\
Durante lo svolgimento del progetto sono stati definiti dei test di accettazione che l'applicazione deve passare per essere ritenuta conclusa.\\
Di seguito vengono elencati i test di accettazione per il front end del progetto. Il codice dei test è così strutturato TA[numero], dove:
\begin{itemize}
    \item \textbf{TA}: test di accettazione;
    \item \textbf{[numero]}: indica un numero progressivo per identificare univocamente il test.
\end{itemize}

\begin{longtable}[c]{|l|p{9cm}|l|}
\caption{Test di accettazione}
\label{tab:test-accettazione}
\\ \hline
\rowcolor{gray!40}
\textbf{ID} &
\textbf{Descrizione} &
\textbf{Stato} \\ \hline
\endhead

TA1 & La landing page deve ricevere le informazioni dell'ordine appena creato nell'e-commerce & \checkmark \\ \hline

TA2 & La landing page deve creare una transazione corrispondente all'ordine creato nell'e-commerce & \checkmark \\ \hline

TA3 & Una volta confermata la transazione, la landing page deve poter reindirizzare l'utente alla lista dei suoi ordini & \checkmark \\ \hline

TA4 & L'applicazione deve mostrare la lista degli ordini correlati al wallet attualmente connesso a MetaMask & \checkmark \\ \hline

TA5 & L'utente deve poter filtrare gli ordini per stato, per indirizzo venditore o per entrambi & \checkmark \\ \hline

TA6 & L'utente deve poter richiedere il reso solo degli ordini in stato \textit{Created}, \textit{Shipped} e \textit{Confirmed} & \checkmark \\ \hline

\end{longtable}

L'ultima settimana di stage è stata organizzata una presentazione finale dell'applicazione web per collaudare tutte le funzionalità richieste e verificare che passi i test di accettazione. Alla fine della presentazione è stata effettuata anche una revisione del codice sorgente insieme al tutor aziendale, superata con successo.