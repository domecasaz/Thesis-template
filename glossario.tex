
%**************************************************************
% Acronimi
%**************************************************************
\renewcommand{\acronymname}{Acronimi e abbreviazioni}

\newacronym[description={\glslink{apig}{Application Program Interface}}]
    {api}{API}{Application Program Interface}

\newacronym[description={\glslink{umlg}{Unified Modeling Language}}]
    {uml}{UML}{Unified Modeling Language}
    
\newacronym[description={\glslink{cicdg}{Continuous Integration/Continuous Delivery}}]
    {cicd}{CI/CD}{Continuous Integration/Continuous Delivery}
    
\newacronym[description={\glslink{waiariag}{Web Accessibility Initiative - Accessible Rich Internet Applications}}]
    {WAI-ARIA}{WAI-ARIA}{Web Accessibility Initiative - Accessible Rich Internet Applications}

%**************************************************************
% Glossario
%**************************************************************
%\renewcommand{\glossaryname}{Glossario}

\newglossaryentry{apig}
{
    name=\glslink{api}{API},
    text=Application Program Interface,
    sort=api,
    description={in informatica con il termine \emph{Application Programming Interface API} (ing. interfaccia di programmazione di un'applicazione) si indica ogni insieme di procedure disponibili al programmatore, di solito raggruppate a formare un set di strumenti specifici per l'espletamento di un determinato compito all'interno di un certo programma. La finalità è ottenere un'astrazione, di solito tra l'hardware e il programmatore o tra software a basso e quello ad alto livello semplificando così il lavoro di programmazione}
}

\newglossaryentry{umlg}
{
    name=\glslink{uml}{UML},
    text=UML,
    sort=uml,
    description={in ingegneria del software \emph{UML, Unified Modeling Language} (ing. linguaggio di modellazione unificato) è un linguaggio di modellazione e specifica basato sul paradigma object-oriented. L'\emph{UML} svolge un'importantissima funzione di ``lingua franca'' nella comunità della progettazione e programmazione a oggetti. Gran parte della letteratura di settore usa tale linguaggio per descrivere soluzioni analitiche e progettuali in modo sintetico e comprensibile a un vasto pubblico}
}

\newglossaryentry{angularg}
{
    name={Angular},
    text=Angular,
    sort=angular,
    description={\emph{Angular} è un framework open source per lo sviluppo di applicazioni web con licenza MIT, evoluzione di AngularJS, sviluppato principalmente da Google}
}

\newglossaryentry{ercg}
{
    name={ERC20},
    text=ERC20,
    sort=erc,
    description={\emph{ERC20} è lo standard tecnico dietro gli smart contract che servono per l'implementazione dei token sulla blockchain di Ethereum. Definisce una struttura dati comune a tutti i token.}
}

\newglossaryentry{metamaskg}
{
    name={MetaMask},
    text=MetaMask,
    sort=metamask,
    description={\emph{MetaMask} è un'estensione browser scaricabile e che permette la connessione del proprio browser alla piattaforma Ethereum e alle applicazioni decentralizzate basate su di essa; consente inoltre di accedere a wallet Ethereum per lo scambio di token ERC20.}
}

\newglossaryentry{cicdg}
{
    name=\glslink{cicd}{Continuous Integration/Continuous Delivery (CI/CD)},
    text=Continuous Integration/Continuous Delivery (CI/CD),
    sort=cicd,
    description={\emph{Continuous Integration/Continuous Delivery} è un metodo per la distribuzione frequente delle applicazioni ai clienti, che prevede l’introduzione dell’automazione nelle fasi di sviluppo dell’applicazione. Principalmente, si basa sui concetti di integrazione, distribuzione e deployment continui.}
}

\newglossaryentry{waiariag}
{
    name=\glslink{WAI-ARIA}{Web Accessibility Initiative - Accessible Rich Internet Applications (WAI-ARIA)},
    text=Web Accessibility Initiative - Accessible Rich Internet Applications(WAI-ARIA),
    sort=waiaria,
    description={\emph{WAI-ARIA}  descrive come aggiungere una semantica o altri metadati al contenuto HTML allo scopo di rendere i controlli lato utente e i contenuti dinamici più accessibili.}
}