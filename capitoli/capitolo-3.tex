% !TEX encoding = UTF-8
% !TEX TS-program = pdflatex
% !TEX root = ../tesi.tex

%**************************************************************
\chapter{Analisi dei requisiti}
\label{cap:analisi-requisiti}
%**************************************************************

\intro{In questo capitolo vengono descritte le funzionalità che il prodotto deve offrire elencando i casi d’uso, gli attori del sistema e i requisiti individuati.}\\

%**************************************************************
\section{Casi d'uso}

\subsection{Attori principali}
Durante la fase di analisi sono stati individuati i seguenti attori principali:
\begin{itemize}
    \item Utente non riconosciuto: attore che indica un utente che non ha ancora eseguito l'autenticazione tramite \gls{metamaskg}\glsfirstoccur;
    \item Utente riconosciuto: attore che indica un utente che ha eseguito l'autenticazione tramite MetaMask e può accedere creazione di un ordine o vedere le informazioni relative agli ordini.
\end{itemize}

\subsection{Attori secondari}
Durante la fase di analisi sono stati individuati i seguenti attori secondari:
\begin{itemize}
    \item MetaMask: plugin esterno al sistema che gestisce i wallet degli utenti riconosciuti e permette loro di essere riconosciuti e accedere all'applicazione web.
\end{itemize}

\subsection{Ordini}
Ogni ordine è caratterizzato da:
\begin{itemize}
    \item un numero identificativo;
    \item indirizzo dell'acquirente;
    \item indirizzo del venditore;
    \item prezzo;
    \item stato dell'ordine.
\end{itemize}

Di seguito è riportata la tabella che descrive i diversi tipi di stato che un ordine può assumere durante il suo ciclo di vita.
\rowcolors{2}{gray!25}{white}
\begin{longtable}[c]{|l|p{7cm}|p{3cm}|}
\hline
\rowcolor{gray!40}
\textbf{Stato} & \textbf{Descrizione} & \textbf{Precondizione}  \\ \hline
\endhead

Created & L'acquirente ha versato la quantità di token richiesti nello smart contract e l'ordine è stato creato & Nessuna \\ \hline
Shipped & Il venditore ha inviato il pacco all'acquirente & Created \\ \hline
Confirmed & L'acquirente ha ricevuto il pacco e ha scansionato il QR code & Shipped \\ \hline
Deleted & Il venditore ha annullato l'ordine & Created/Shipped \\ \hline
Refund Asked & L'acquirente ha chiesto il rimborso dell'ordine & Created/Shipped/ Confirmed \\ \hline
Refunded & Il venditore ha confermato la richiesta di rimborso & Refund Asked \\ \hline
\end{longtable}

\subsection{Elenco casi d'uso}
Per lo studio dei casi di utilizzo del prodotto sono stati creati dei diagrammi.
I diagrammi dei casi d'uso (in inglese \emph{Use Case Diagram}) sono diagrammi di tipo \gls{uml}\glsfirstoccur dedicati alla descrizione delle funzioni o servizi offerti da un sistema, così come sono percepiti e utilizzati dagli attori che interagiscono col sistema stesso.

\subsection{Landing Page}
\begin{figure}[!h] 
    \centering 
    \includegraphics[width=1\columnwidth]{immagini/usecase/UCLandingPage.png} 
    \caption{Use Cases - Landing Page}
\end{figure}

\begin{usecase}{1}{Connessione al wallet}
\usecaseactorsP{Utente non riconosciuto}
\usecaseactorsS{MetaMask}
\usecasepre{L'Utente non ha effettuato il collegamento al proprio wallet}
\usecasepost{L'Utente si è identificato e può proseguire con il pagamento}
\usecasedesc{L'Utente vuole interagire con lo smart contract di ShopChain, quindi deve collegare il wallet ad esso per poter autorizzare successivamente le transazioni.
\begin{enumerate}
    \item L'Utente clicca su “Connetti wallet”;
    \item Si apre il pop-up di MetaMask dove egli inserisce i dati;
    \item L’utente dà il permesso allo smart contract di interagire con il proprio wallet.
\end{enumerate}
}
\label{uc:connessione-wallet}
\end{usecase}

\begin{usecase}{2}{Visualizzazione Dati Transazione}
\usecaseactorsP{Utente riconosciuto}
\usecasepre{L'Utente ha iniziato la fase di pagamento nella piattaforma e-commerce e ha eseguito l’accesso al wallet}
\usecasepost{L'Utente ha visualizzato i dati della transazione}
\usecasedesc{L'Utente visualizza nella Landing Page:
\begin{enumerate}
    \item l’importo corrispondente al/i prodotto/i selezionato/i;
    \item eventuali fee stimate (medie);
    \item l’indirizzo del venditore (e-commerce).
\end{enumerate}
}
\label{uc:visualizzazione-transazione}
\end{usecase}

\begin{usecase}{3}{Conferma Pagamento}
\usecaseactorsP{Utente riconosciuto}
\usecaseactorsS{MetaMask}
\usecasepre{L'Utente è stato riconosciuto e ha visualizzato i dati della transazione}
\usecasepost{La transazione è stata avviata e le criptovalute si trovano nello smart contract}
\usecasedesc{
\begin{enumerate}
    \item L'Utente autorizza la transazione iniziata;
    \item l’interfaccia di MetaMask permette la conferma;
    \item l’importo viene trasferito dal suo wallet allo smart contract;
    \item viene creato l’ordine nello smart contract in stato "Created".
\end{enumerate}
}
\label{uc:conferma-pagamento}
\end{usecase}

\newpage

\subsection{Applicazione Web}
\begin{figure}[!h] 
    \centering 
    \includegraphics[width=1\columnwidth]{immagini/usecase/UCWebApp.png} 
    \caption{Use Cases - Applicazione Web}
\end{figure}

\begin{usecase}{4}{Visualizzazione Ordini}
\usecaseactorsP{Utente riconosciuto}
\usecasepre{L’Utente è stato riconosciuto tramite la connessione al suo wallet}
\usecasepost{L’Utente ha visualizzato gli ordini relativi al wallet collegato}
\usecasedesc{L’Utente può visualizzare una tabella contenente gli ordini (in cui egli è uno degli attori) con identificativo, indirizzo dell’altro attore, costo, stato.}
\label{uc:visualizzazione-ordini}
\end{usecase}

\begin{figure}[!h] 
    \centering 
    \includegraphics[width=1\columnwidth]{immagini/usecase/UC4.png} 
    \caption{Use Case 4 - Visualizzazione Lista Ordini}
\end{figure}

\begin{usecase}{4.1}{Visualizzazione Singolo Ordine nella Lista}
\usecaseactorsP{Utente riconosciuto}
\usecasepre{L’Utente ha visualizzato la lista degli ordini relativa al suo indirizzo}
\usecasepost{L’Utente ha visualizzato i dati specifici di un ordine}
\usecasedesc{L’Utente può visualizzare i dettagli di un ordine selezionato,
inclusi il log degli stati e le eventuali operazioni disponibili.}
\usecasealt{Se l’Utente prova a visualizzare un ordine al quale non ha accesso, viene visualizzata una pagina di errore.}
\label{uc:visualizzazione-singolo-ordine}
\end{usecase}

\begin{usecase}{6}{Errore Ordine Inaccessibile}
\usecaseactorsP{Utente riconosciuto}
\usecasepre{L’Utente è stato riconosciuto tramite la connessione al suo wallet}
\usecasepost{L’Utente ha visualizzato la pagina d’errore}
\usecasedesc{
\begin{itemize}
    \item L’Utente cerca di visualizzare i dettagli di un ordine inesistente o al quale non ha accesso;
    \item la webApp presenta una pagina d’errore che permette di tornare indietro.
\end{itemize}
}
\label{uc:errore-ordine-inacessibile}
\end{usecase}

\begin{figure}[!h] 
    \centering 
    \includegraphics[width=1\columnwidth]{immagini/usecase/UC4.1.png} 
    \caption{Use Case 4.1 - Visualizzazione Singolo Ordine nella Lista}
\end{figure}

\begin{usecase}{4.1.1}{Visualizzazione ID Ordine}
\usecaseactorsP{Utente riconosciuto}
\usecasepre{L’Utente si trova nella sezione "Visualizzazione Singolo Ordine nella Lista" (UC4.1) e desidera visualizzare l’id dell’ordine}
\usecasepost{L’Utente ha visualizzato l’id dell’ordine}
\usecasedesc{L’Utente sta visualizzando i dettagli di un singolo ordine e tra i campi è presente l’id.}
\label{uc:visualizzazione-id-ordine}
\end{usecase}

\newpage

\begin{usecase}{4.1.2}{Visualizzazione Indirizzo}
\usecaseactorsP{Utente riconosciuto}
\usecasepre{L’Utente si trova nella sezione "Visualizzazione Singolo Ordine nella Lista" (UC4.1) e desidera visualizzare l’indirizzo dell’altro attore dell’ordine}
\usecasepost{L’Utente ha visualizzato l’indirizzo dell’altro attore dell’ordine}
\usecasedesc{L’Utente sta visualizzando i dettagli di un singolo ordine e tra i campi è presente l’indirizzo dell’altro attore dell’ordine (venditore per il compratore e viceversa).}
\label{uc:visualizzazione-indirizzo}
\end{usecase}

\begin{usecase}{4.1.3}{Visualizzazione Ammontare Ordine}
\usecaseactorsP{Utente riconosciuto}
\usecasepre{L’Utente si trova nella sezione "Visualizzazione Singolo Ordine nella Lista" (UC4.1) e desidera visualizzare l’ammontare dell’ordine}
\usecasepost{L’Utente ha visualizzato l’ammontare dell’ordine}
\usecasedesc{L’Utente sta visualizzando i dettagli di un singolo ordine e tra i campi è presente l’ammontare dell’ordine.}
\label{uc:visualizzazione-ammontare-ordine}
\end{usecase}

\begin{usecase}{4.1.4}{Visualizzazione Stato Ordine}
\usecaseactorsP{Utente riconosciuto}
\usecasepre{L’Utente si trova nella sezione "Visualizzazione Singolo Ordine nella Lista" (UC4.1) e desidera visualizzare lo stato dell’ordine}
\usecasepost{L’Utente ha visualizzato lo stato dell’ordine}
\usecasedesc{L’Utente sta visualizzando i dettagli di un singolo ordine e tra i campi è presente lo stato in cui si trova l’ordine.}
\label{uc:visualizzazione-stato-ordine}
\end{usecase}

\newpage

\begin{usecase}{5}{Filtra Ordini}
\usecaseactorsP{Utente riconosciuto}
\usecasepre{L’Utente è stato riconosciuto tramite la connessione al suo wallet}
\usecasepost{L’Utente ha visualizzato i dati relativi ad ordini che corrispondono ai requisiti di ricerca}
\usecasedesc{L’Utente può visualizzare in modo più ordinato le transazioni
alle quali è interessato attraverso dei filtri.}
\label{uc:filtra-ordini}
\end{usecase}

\begin{figure}[!h] 
    \centering 
    \includegraphics[width=1\columnwidth]{immagini/usecase/UC5.png} 
    \caption{Use Case 5 -Filtra Ordini}
\end{figure}

\begin{usecase}{5.1}{Filtra per Stato}
\usecaseactorsP{Utente riconosciuto}
\usecasepre{L’Utente è stato riconosciuto tramite la connessione al suo wallet}
\usecasepost{L’Utente ha visualizzato i dati relativi ad ordini che corrispondono ai requisiti di ricerca}
\usecasedesc{
\begin{itemize}
    \item L’Utente seleziona lo stato degli ordini che vuole visualizzare;
    \item l’Utente clicca sul bottone "Applica Filtri".
\end{itemize}
}
\usecasealt{Se l’Utente applica un filtro che non viene rispettato da nessun ordine, viene visualizzato un errore.}
\label{uc:filtra-ordini-stato}
\end{usecase}

\begin{usecase}{5.2}{Filtra per Indirizzo}
\usecaseactorsP{Utente riconosciuto}
\usecasepre{L’Utente è stato riconosciuto tramite la connessione al suo wallet}
\usecasepost{L’Utente ha visualizzato i dati relativi ad ordini che corrispondono ai requisiti di ricerca}
\usecasedesc{
\begin{itemize}
    \item L’Utente inserisce l'indirizzo degli ordini che vuole visualizzare;
    \item l’Utente clicca sul bottone "Applica Filtri".
\end{itemize}
}
\usecasealt{Se l’Utente applica un filtro che non viene rispettato da nessun ordine, viene visualizzato un errore.}
\label{uc:filtra-ordini-inidirizzo}
\end{usecase}

\begin{usecase}{5.3}{Filtra per Indirizzo e Stato}
\usecaseactorsP{Utente riconosciuto}
\usecasepre{L’Utente è stato riconosciuto tramite la connessione al suo wallet}
\usecasepost{L’Utente ha visualizzato i dati relativi ad ordini che corrispondono ai requisiti di ricerca}
\usecasedesc{
\begin{itemize}
    \item L’Utente inserisce l'indirizzo e seleziona lo stato degli ordini che vuole visualizzare;
    \item l’Utente clicca sul bottone "Applica Filtri".
\end{itemize}
}
\usecasealt{Se l’Utente applica un filtro che non viene rispettato da nessun ordine, viene visualizzato un errore.}
\label{uc:filtra-ordini-stato-indirizzo}
\end{usecase}

\begin{usecase}{8}{Errore nessun ordine rispetta le condizioni del filtro}
\usecaseactorsP{Utente riconosciuto}
\usecasepre{L’Utente ha già applicato un filtro le cui condizioni non sono rispettate da nessun ordine}
\usecasepost{L’Utente ha visualizzato l’errore e può provare ad applicare un altro filtro}
\usecasedesc{
\begin{itemize}
    \item L’Utente applica un filtro a cui non corrisponde nessun ordine;
    \item viene mostrato un messaggio di errore e la lista di tutti gli ordini dell’Utente.
\end{itemize}
}
\label{uc:errore-filtri}
\end{usecase}

\begin{usecase}{7}{Visualizzazione Dettagli Ordine}
\usecaseactorsP{Utente riconosciuto}
\usecaseactorsS{MetaMask}
\usecasepre{L’Utente è stato riconosciuto tramite la connessione al suo wallet}
\usecasepost{L'Utente si trova nella pagina di visualizzazione dettagli del singolo ordine}
\usecasedesc{
\begin{itemize}
    \item L’Utente clicca sul bottone "Vedi dettagli";
    \item l’Utente viene reindirizzato alla pagina di visualizzazione dettagli del singolo ordine.
\end{itemize}
}
\label{uc:visualizzazione-dettagli-ordine}
\end{usecase}

\newpage

\begin{figure}[!h] 
    \centering 
    \includegraphics[width=1\columnwidth]{immagini/usecase/UC7.png} 
    \caption{Use Case 7 - Visualizzazione Dettagli Ordine}
\end{figure}

\begin{usecase}{7.1}{Chiedi Rimborso}
\usecaseactorsP{Utente riconosciuto}
\usecaseactorsS{MetaMask}
\usecasepre{L'Utente si trova nella sezione "Visualizza Dettagli Singolo Ordine" (UC18) e desidera annullare uno degli acquisti sul proprio e-commerce}
\usecasepost{L'Utente ha richiesto il reso}
\usecasedesc{
\begin{itemize}
    \item L'Utente clicca sul bottone "Chiedi Reso";
    \item viene aperto il pop-up di MetaMask che richiede la conferma di una transazione con fee minima affinché venga emessa la richiesta di rimborso;
    \item lo stato dell’ordine passa da "Confirmed" a "Refund Asked".
\end{itemize}
}
\label{uc:chiedi-rimborso}
\end{usecase}

\begin{usecase}{7.2}{Visualizza Log di Cambio Stato}
\usecaseactorsP{Utente riconosciuto}
\usecasepre{L'Utente si trova nella sezione "Visualizza Dettagli Singolo Ordine" (UC18) e desidera visualizzare il log di cambio stato dell'ordine}
\usecasepost{L’Utente ha visualizzato il log di cambio stato dell'ordine}
\usecasedesc{L’Utente sta visualizzando i dettagli di un singolo ordine e sulla pagina è presente il log di cambio stati dell'ordine in cui è presente lo stato dell'ordine e il timestamp di quando l'ordine ha cambiato stato.}
\label{uc:visualizzazione-log}
\end{usecase}

\newpage

%**************************************************************
\section{Tracciamento dei requisiti}

Durante la fase di analisi dei requisiti e di individuazione degli use cases è stata stilata una tabella che tracci i requisiti in rapporto agli use cases. Sono stati individuati diversi tipi di requisiti e si è quindi fatto utilizzo di un codice identificativo per distinguerli.\\
Il codice dei requisiti è così strutturato R[F/Q][N/D] dove:
\begin{itemize}
    \item[] R = Requisito;
    \item[] F = Funzionale;
    \item[] Q = Qualitativo;
    \item[] N = Necessario (obbligatorio);
    \item[] D = Desiderevole;
\end{itemize}

Le fonti dei requisiti possono essere le seguenti:
\begin{itemize}
    \item[] UC[numero]: indica un caso d'uso;
    \item[] Committente: indica il capo del progetto.
\end{itemize}

Nelle tabelle \ref{tab:requisiti-funzionali}, \ref{tab:requisiti-qualitativi} e sono riassunti i requisiti e il loro tracciamento con gli use case delineati in fase di analisi.

\rowcolors{2}{gray!25}{white}

\begin{longtable}[c]{|l|p{8.4cm}|l|}
\caption{Tabella del tracciamento dei requisti funzionali}
\label{tab:requisiti-funzionali}
\\ \hline
\rowcolor{gray!40}
\textbf{Codice} &
\textbf{Descrizione} &
\textbf{Fonti} \\ \hline
\endhead

\textbf{RFN 1} & Landing page che permetta di completare il pagamento & UC1 \\ \hline

\textbf{RFN 1.1} & L’utente deve potersi connettere al proprio wallet tramite MetaMask & UC1 \\ \hline

\textbf{RFN 1.2} & L’utente deve poter visualizzare l’importo totale della transazione & UC2 \\ \hline

\textbf{RFN 1.3} & L’utente deve poter visualizzare l’address del venditore & UC2 \\ \hline

\textbf{RFN 1.4} & L’utente deve poter autorizzare la transazione, completando l’acquisto del prodotto & UC3 \\ \hline

\textbf{RFN 2} & WebApp che permetta di visualizzare ed interagire con i propri ordini & UC4 \\ \hline

\textbf{RFN 2.1} & L’utente deve potersi connettere al proprio wallet tramite MetaMask & UC1 \\ \hline

\textbf{RFN 2.2} & L’utente deve poter visualizzare tutte le transazioni effettuate con ShopChain dai suoi indirizzi wallet collegati & UC4 \\ \hline

\textbf{RFN 2.3} & L’utente deve poter filtrare le transazioni. & UC5 \\ \hline

\textbf{RFN 2.3.1} & L’utente deve poter filtrare le transazioni in base al loro stato & UC5.1 \\ \hline

\textbf{RFN 2.3.2} & L’utente deve poter filtrare le transazioni in base all’indirizzo & UC5.2 \\ \hline

\textbf{RFN 2.3.3} & L’utente deve poter filtrare le transazioni in base al loro stato e all'indirizzo & UC5.3 \\ \hline

\textbf{RFN 2.4} & L’utente deve poter visualizzare un singolo ordine della lista & UC4.1 \\ \hline

\textbf{RFN 2.4.1} & L’utente deve poter l'id di un singolo ordine & UC4.1.1 \\ \hline

\textbf{RFN 2.4.2} & L’utente deve poter visualizzare l'indirizzo del venditore di un singolo ordine & UC4.1.2 \\ \hline

\textbf{RFN 2.4.3} & L’utente deve poter visualizzare il prezzo di un singolo ordine & UC4.1.3 \\ \hline

\textbf{RFN 2.4.4} & L’utente deve poter visualizzare lo stato di un singolo ordine & UC4.1.4 \\ \hline

\textbf{RFN 2.5} & L’utente deve poter chiedere il rimborso di un ordine & UC7.1 \\ \hline

\textbf{RFN 2.6} & L’utente deve poter visualizzare il log di cambio stati di un ordine & UC7.2 \\ \hline

\end{longtable}

\begin{longtable}[c]{|l|p{8.4cm}|l|}
\caption{Tabella del tracciamento dei requisti qualitativi}
\label{tab:requisiti-qualitativi}
\\ \hline
\rowcolor{gray!40}
\textbf{Codice} &
\textbf{Descrizione} &
\textbf{Fonti} \\ \hline
\endhead

\textbf{RQN 1} & Deve essere fornito un documento tecnico che descriva le maschere & Committente \\ \hline

\textbf{RQN 2} & Il codice deve essere condiviso su una repository pubblica & Committente\\ \hline

\textbf{RQD 3} & L’applicazione web deve essere accessibile a utenti con disabilità & Committente\\ \hline

\textbf{RQD 4} & L’applicazione web deve essere responsive & Committente\\ \hline

\end{longtable}