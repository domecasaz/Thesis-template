    % !TEX encoding = UTF-8
% !TEX TS-program = pdflatex
% !TEX root = ../tesi.tex

%**************************************************************
\chapter{Conclusioni}
\label{cap:conclusioni}

%**************************************************************
\section{Raggiungimento degli obiettivi}
Per quanto riguarda gli obiettivi prefissati all'inizio dello stage (sezione \ref{sec:requisiti-obiettivi}) risultano tutti raggiunti. Di seguito la tabella riassuntiva.

\begin{longtable}[c]{|p{3cm} p{3cm}|}
\caption{Riepilogo obiettivi tirocinio}
\label{tab:obiettivi-tirocinio}
\\ \hline
\rowcolor{gray!40}
\textbf{Obiettivo} &
\textbf{Stato} \\ \hline
\endhead

O01 & Soddisfatto \\ \hline

O02 & Soddisfatto \\ \hline

O03 & Soddisfatto \\ \hline

D01 & Soddisfatto \\ \hline

F01 & Soddisfatto \\ \hline

\end{longtable}

%**************************************************************
\section{Conoscenze acquisite}
Grazie a questo percorso di tirocinio si è potuto approfondire e apprendere un gran numero di argomenti che riguardano le applicazione web decentralizzate, ovvero basate su blockchain. Oltre a ciò, l'esperienza di stage è stata utile per apprendere un metodo lavorativo reale, derivante dal modus operandi dell'azienda. Di seguito vengono riportate le principali conoscenze a livello professionale che ho acquisito con questo percorso.

\paragraph{\textbf{Nuovi linguaggi di programmazione e framework}}
All'inizio dello stage si è avuto modo di studiare e approfondire il framework \textbf{Angular}, nonché tutti gli strumenti ausiliari per la gestione grafica (\textit{Angular Material} e \textit{Bootstrap}). Con l'apprendimento del linguaggio \textbf{TypeScript} e l'approfondimento di \textbf{JavaScript}, mi sono ulteriormente avvicinato allo sviluppo di applicazioni web, scoprendo nuove funzionalità di questi linguaggi, quali gli \textit{Observables} per la gestione degli eventi.

\paragraph{\textbf{Nuove competenze progettuali}}
Attraverso l'analisi e la progettazione delle maschere si sono messe in pratica le mie conoscenze per la realizzazione di componenti e servizi, facendo uso di opportuni \textbf{design patterns} supportati dal framework utilizzato per la codifica, in modo da ottimizzare al meglio l'applicazione web.

\paragraph{\textbf{Nuovi strumenti di collaborazione}}
Ho appreso in buona parte il funzionamento di repository \textit{Git}, in particolare utilizzando \textit{GitHub} per il versionamento del codice e la collaborazione con gli altri membri che lavoravano al progetto. Ho imparato, inoltre, ad utilizzare uno strumento per la gestione delle attività, ovvero \textit{Trello}.

\paragraph{\textbf{Apprendimento del metodo Scrum}}
Nel corso del tirocinio ho appreso il metodo \textbf{Scrum} utilizzato dall'azienda e ho imparato principalmente a suddividere e organizzare le mie attività e i miei obiettivi affinché rientrassero nel tempo a mia disposizione. Ho fruito a pieno di tutte le risorse a mia disposizione e ho apprezzato i meeting settimanali di allineamento sullo svolgimento del progetto.

\paragraph{\textbf{Approfondimento tecnologia blockchain}}
Durante lo svolgimento del progetto ho avuto modo di confrontarmi con colleghi più esperti di me in ambito blockchain e, grazie al confronto con loro, ho potuto approfondire le mie conoscenze su questa tecnologia.

%**************************************************************
\section{Valutazione personale}

Nel corso del tirocinio ho maturato diverse idee riguardanti il rapporto tra il mondo universitario e il mondo del lavoro. Buona parte delle conoscenze del mondo universitario sono in gran parte essenziali, perché mi hanno aiutato ad avere un'impostazione mentale e uno spirito critico sul mio operato, ma alcuni degli argomenti trattati mi sono apparsi troppo vecchi rispetto alle necessità del mercato attuale. Questa esperienza lavorativa mi ha aiutato a capire la complementarietà tra università e lavoro. Nel mondo del lavoro manca la rigidità organizzativa e metodica tipica dell'università. Tuttavia però, il mondo lavorativo è molto più dinamico e in continua evoluzione rispetto al mondo universitario, che rimane purtroppo bloccato in un pensiero più antico che può portare gli studenti a perdere interesse per certi corsi di studio che non stanno al passo con i tempi.\\
Questa attività di tirocinio, secondo me, è stata molto utile per avvicinarsi all'ambiente lavorativo reale e sono stato molto soddisfatto dell'azienda che mi ha ospitato e dei colleghi con cui ho avuto modo di collaborare. La loro disponibilità e quella del tutor aziendale è stata per me essenziale per avere dei punti di riferimento con cui delimitare il mio percorso di tirocinio.\\
Per quanto riguarda il percorso universitario, invece, ritengo che si siano affrontati argomenti fondamentali per formarsi ed inserirsi nel mondo del lavoro. La maggior parte delle lezioni che ho frequentato mi ha aiutato ad imparare come affrontare un progetto di gruppo e come risolvere le difficoltà. In aggiunta, le conoscenze teoriche e pratiche di progettazione e codifica, acquisite con i corsi universitari, sono state sufficienti nel progetto di stage per poter operare su nuovi linguaggi di programmazione e su framework sconosciuti.\\
La parte più carente che ho trovato nell'offerta formativa universitaria riguarda la poca formazione sull'utilizzo di strumenti di configurazione e di collaborazione. In particolare, faccio riferimento ad uno strumento molto famoso, come \textit{Git}, che può essere adottato trasversalmente a varie discipline e che è molto utilizzato nel mondo del lavoro. Questa mancanza è purtroppo colmata solo da alcuni corsi a scelta o dallo studio autonomo dello studente, che però difficilmente riesce a conciliarlo con le ore di studio richieste dai corsi universitari. Una seconda carenza l'ho ritrovata nella poca modernità di alcuni contenuti riportati in alcuni corsi, a discapito dello studio di tecnologie e argomenti più moderni e attuali. Questo rischia di lasciare il mondo universitario bloccato a tecnologie antiche e non più in uso, e che quindi diventa quasi inutile studiarle se non per fini storici.\\
Concludendo questa esperienza di tirocinio per me è stata molto formativa ed utile a livello personale non solo perché mi ha fatto capire i punti di forza e i punti di debolezza del mio carattere e come valorizzarli al meglio, ma anche perché mi ha aiutato a capire meglio quali possono essere i miei ambiti e obiettivi a livello lavorativo.